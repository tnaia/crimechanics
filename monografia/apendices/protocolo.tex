\chapter{Protocolo de comunicação entre C++ e Java}\label{ap:protocolo}

A necessidade de comunicação entre os subsistemas de processamento da inteligência articial implementada e os mecanismos de controle e renderização do jogo surgiram tão logo se foram definidos dois requisitos do projeto, em seu início: o emprego de C++, de modo a refletir um padrão amplamente empregado na indústria de jogos, e a opção pelo uso do interpretador de AgentSpeak, Jason, que possui apenas interfaces para Java.

Definiu-se então um protocolo simples, operando via TCP (\emph{loopback}), o que não incorre em perdas significativas de eficiência na troca de informações. O protocolo, que passou por diversas modificações ao longo da evolução do projeto, baseia-se na troca de mensagens, sequencias de inteiros, que respeitam regras simples:
\begin{enumerate}
\item o primeiro inteiro enviado representa quantos inteiros mais compõe a mensagem;
\item o segundo inteiro identifica o nome da mensagem;
\item os demais inteiros são parâmetros da mensagem.
\end{enumerate}

O nome da mensagem é recuperado pelo receptor da mensagem pela leitura de um arquivo de referência --- é a linha de número igual ao inteiro recebido. Munido dessa cadeia de caracteres, o receptor então pode buscar em um mapa interno a função responsável por tratar a mensagem e seus argumentos. De modo análogo, os argumentos representam, na maior parte das vezes, o número da linha que deve ser lida em algum arquivo do sistema (pré-definido para cada nome de mensagem) contendo algum complemento da informação.

\section{Arquivos consultados}

Os arquivos que são lidos pelas rotinas de tratamento das mensagens são
\begin{description}
\item[estímulos] contém estímulos que podem ser enviados a agentes no decorrer de diálogos;
\item[respostas] contém as respostas que agentes podem expressar em diálogos;
\item[capangas] contém nome (e possivelmente outras informações) a respeito de capangas na equipe dos \nomeGrupo;
\item[lugares] contém o nome de lugares passíveis de roubo;
\item[predicados] contém uma lista de avisos que podem ser dados à polícia;
\item[configurações] contém, respectivamente, em suas linhas
\begin{enumerate}
\item o número da porta de comunicação loopback empregado pelos subsistemas em sua comunicação;
\item o número de slots que o jogo possui
\item o maior nível de suspeita possível para capangas (que é também o maior nível de segurança para lugares)
\end{enumerate}
\end{description}

\section{As mensagens}

O conjunto de mensagens passou por várias alterações desde sua primeira concepção, o que não somente era esperado, como demonstrou a utilidade da arquitetura de troca de mensagens que havia sido projetada, preparada para acomodar sem grande dificuldade adições e remoções de mensagens. As mensagens são listadas a seguir.

\begin{enumerate}\footnotesize
\item \verb!iniciaDialogo( )!
\item \verb!retornaIdAgente()!
%
\item \verb!estimuloagente ( # agente, # estímulo )!
\item \verb!atualizaBeliefs(# agente, # estimulo)!
%
\item \verb!ofereceEscolhas( # agente, # reacao1, # reacao2, … )!
\item \verb!returnEscolha(# agente, #escolha)!
%
%
\item \verb!fimDialogo( #agente )!
\item \verb!acaoDaPolicia( #acao, #alvo )!
%
%ações policiais:       1 aumentar nivel suspeita - P
            %2 diminuir nivel suspeita - P

%3 prender - P

%4 aumentar nivel segurança - L

%5 diminuir nivel segurança - L
%
\item \verb!avisaPolicia( #sujeito, #predicado )!
\item \verb!avisaPolicia( #sujeito, #predicado, #alvo )!
%
%predicados:
%1 P fez-compra-suspeita
%2 P agiu-de-modo-suspeito-em L
%3 L foi-roubado-com-sucesso
%4 L foi-roubado-sem-sucesso
%legenda: (P - pessoa; L - lugar)   
%
%

%[10 escreveBlackBoard(#sujeito, #predicado)] => 8
%
\item \verb!loadGame(! \#slot )
\item \verb!newGame(! \#slot \verb!)!
\item \verb!saveGame(! \#slot \verb!)!
\item \verb!quitGame()!
\end{enumerate}

0 - “número de números”
1 - tipo de mensagem
2 … - parâmetros

EXEMPLOS DE ARQUIVOS DE CONSULTA

capangas:
joe
josé
ricardo
arroz
roney
antônio

estímulos:
oferece-propina
fala-amigavel
elogio
insulto
fala-timida

reações:
aceita-propina
rejeita-propina
insultar
elogiar
agradecer

lugares:
oficina mecânica (QG),
mercado,
loja de importados,
loja de carros,
banco, e
joalheria.

configurações:




