\chapter{Implementação}

Neste capítulo descrevemos brevemente o sistema, abordando a organização de seus arquivos componentes e a estratégia de organização do software tendo em vista a criação de uma estrutura de fácil manutenção e expansão.

\section{Visão global}

O jogo é disparado por um pequeno aplicativo cuja única função é iniciar\footnote{Isso foi feito por meio do fork to processo e alteração de suas imagens nos processos pai e filho resultantes.} os componentes Java e C++ do projeto, que passam então a se comunicar via TCP (via \emph{loopback}), empregando a porta de número especificado no arquivo de configuração.

Passada essa etapa de estabelecimento de comunicação, a interação entre C++ e Java se dá pela troca de mensagens, segundo um protocolo simples desenvolvido no trabalho para este projeto em particular. Sua descrição consta no apêndice~\ref{ap:protocolo}. As responsabilidades ficam então assim divididas: a renderização e gerenciamento dos arquivos de diálogos e variáveis do jogo (inventário, membros ativos da gangue, recursos financeiros, etc.) fica a encargo do programa C++, enquanto que toda a tarefa de simulação da arquitetura  BDI e da gerência do blackboard, assim como manutenção do registro dos estados psicológicos, personalidade e conduta dos agentes são de responsabilidade do programa Java (o que inclui o interpretador de AgentSpeak, Jason). Ademais, na arquitetura implementada, o processo Java roda como cliente TCP, e o C++ como servidor.

%Como as coisas funcionam. O que faz o quê (macroscopicamente).

%Uma explicação sucinta da estrutura do programa, como ele inicia
%(forking) e como age em regime (responsabilidades java vs C++, e o
%protocolo).

\section{Compromissos (\emph{trade-offs})}\label{sec:tradeoffs}

Esta seção descreve algumas opções que o grupo fez, sacrificando certos atributos desejáveis em prol de outros no projeto. 

Já foi mencionado que a opção por uma licensa livre para o projeto incorreu tanto em beneífios como em perdas para o projeto. Passou-se a dispor de espaço para hospedagem do projeto, por um lado, embora, por outro, tenha sido necessário abrir mão do uso de engines e bibliotecas proprietárias para o desenvolvimento.

Outro ponto de compromisso foi a decisão por um modelo ``fechado'' de respostas de agentes nos diálogos: a reação dos agentes nesse caso ficou restrita ao conjunto de opções previstas pelo escritor do diálogo. Se por um lado perde-se com a restrição da manifestação do BDI a um conjunto pré-modelado de respostas, por outro ganha-se em condução do diálogo, uma vantagem que não pode ser menosprezada no design do jogo.

Por fim, devemos mencionar que, dado o escopo temporal do projeto, optamos por simplificações na sofisticação da renderização. O ganho de tempo dedicado ao projeto veio, nesse caso, às custas de um distanciamento do objeto-alvo de estudo, jogos comerciais, uma vez que o investimento n desenvolvimento de uma apresentação gráfica adequada é uma preocupação constante nesse meio.

\section{Arquitetura}

Conforme descrito anteriormente, a arquitetura do sistema é baseada em dois grandes subsistemas. O subsistema implementado em Java que inclui os códigos interpretados pelo Jason e o subsistema implementado em C++. A figura \ref{arquiteturaGeral} ilustra esta situação.

\begin{figure}
\centering
\includegraphics{figuras/arquitetura.jpg}
\caption{Visão geral da arquitetura do sistema}
\label{arquiteturaGeral}
\end{figure}


\subsection{Subsistema: Java}

A figura \ref{arquiteturaJava} apresenta a arquitetura do subsistema implementado em Java. Existem três classes principais, \emph{Manipuladora}, \emph{Comunicadora} e \emph{Level}.
A classe Manipuladora é responsável por consultar e manter atualizados os arquivos de modelos de agentes (explicados em \ref{estruturaPastas}), já a classe Comunicadora é a responsável por receber, enviar e traduzi as mensagens que chegam através da conexão com o subsitema C++. Ambas as classes se comunicam com a classe Level que nada mais é do que a classe de ambiente dos agentes. A classe Level é uma classe utilizada pelo Jason, ela representa o ambiente (enviroment) e é nela que todas as ações dos agentes foram implementadas, no caso deste programa as ações dos agentes são basicamente escolher um tipo de resposta ou atualizar variáveis de controle dos agentes e dos lugares.

\begin{figure}
\centering
\includegraphics[height=10cm]{figuras/arquitetura-java.jpg}
\caption{Visão geral da arquitetura do subsistema Java}
\label{arquiteturaJava}
\end{figure}

\subsubsection{Arquitetura BDI}
Após pesquisas, o grupo encontrou um \emph{plug-in} do interpretador Jason para rodar com a IDE eclipse, com isto o trabalho ficou mais simples, todo a parte do projeto que utilizava linguagem Java e AgentSpeak foi implementada utilizando a IDE eclipse.
A arquitetura dos agentes não ficou complexa, basicamente todos possuem caracterísicas em comum como a \emph{personalidade} e o \emph{estado psicológico}. No caso de um diálogo estas duas características são necessárias pois é baseado nelas que o agente toma decisões, a atualização destas características ocorre da seguinte forma:
\begin{enumerate} 
\item Classe comunicadora recebe mensagem do tipo estimuloagente ( num agente, num estímulo )
\item Classe Level recebe os parâmetros e insere a crença estímulo(nomeEstimulo) no agente em questão
\item Agente atualiza sua personalidade e seu estado psicológico
\end{enumerate}

Caso o agente receba opções de resposta, ele avalia o(s) estímulo(s) recebido pela fala do jogador e procura a resposta que contenha a reação que lhe interessa, este processo acontece da seguinte forma:

\begin{enumerate} 
\item Classe comunicadora recebe mensagem do tipo estimuloagente ( num agente, num estímulo )
\item Classe Level recebe os parâmetros e insere a crença estímulo(nomeEstimulo) no agente em questão
\item Agente atualiza sua personalidade e seu estado psicológico
\item Classe comunicadora recebe mensagem do tipo ofereceEscolhas( num agente, num reacao1, num reacao2, ... )
\item Classe Level recebe os parâmetros e insere a crença respostas(reacao1,reacao2) no agente em questão
\item O agente procura em sua lógica qual a melhor reacao segundo sua personalidade e seu estado psicológico 
\item O agente executa a ação responder(reacao) que envia para o subsitema C++ a escolha do agente
\end{enumerate}

Os casos demonstrados acima se aplicam a qualquer agente com o qual o jogador inicie um diálogo, inclusive os capangas, isto foi feito para manter a arquitetura dos agentes mais genérica.

\subsubsection{Blackboard}

A técnica \emph{blackboard} foi implementada dentro da arquitetura dos agentes ocultos da polícia, esta decisão de projeto visou um melhor desempenho do software.
Basicamente foi criado um agente oculto para cada tipo de informação que pode ser enviada por um agente policial, assim, toda vez que um agente policial executar a ação \emph{avisaPolicia(sujeito,predicado)} é adicionada uma crença em cada agente oculto sobre este aviso, no entanto, apenas um deles sabe o que deve fazer com esta informação, este agente processa a informação e avisa todos os outros agentes a nova informação, esta nova informação pode ser uma ação da polícia que será enviada aos agentes policiais ou pode ser uma nova informação que outro agente oculto sabe processar, assim o processo se torna incremental até que a informação seja completamente processada e resulte em algum tipo de ação da polícia.


\subsection{Subsistema: C++}

A concepção do subsistema C++ foi talvez uma das etapas mais longas do
projeto. Isto pois exite toda uma série de questões inerentes à a organização
do controle da lógica e renderização de jogos, e, como é natural em
sistemas complexos, pequenas decisões da estrutura e organização
refletem grandemente na adequação do sistema à manutenção e
alterações.

Manteve-se em mente durante o pojeto que, dada a novidade do campo de
desenvolvimento de jogos para os integrantes do grupo, dever-se-ia
optar por soluções mais afeitas à modificações. Essa opção é natural,
dado que era grande a probabilidade de que fosse neessário efetuar
modificações para comportar necessidades identificadas ao longo do
desenvolvimento. Ademais, a metodologia de design do jogo escolhida
valia-se de prototipações frequentes, e, nesse sentido, foi importante
que o sistema em si refletisse essa flexibilidade, de modo a atender
às requisições de alteração de comportamento exigidas pela
experimentação com a lógica do jogo. Ressaltamos, entretanto, que essa
flexibilidade não foi total, e desde muito cedo o projeto se fixou em
restrições de apresentação gráfica, numa tentativa de se guardar certo
controle sobre o crescimento da complexidade da renderização do jogo.

Posto de maneira simples, o jogador transita no jogo por estados, como
o \emph{menu principal}\footnote{Menu que permite optar por iniciar um
  novo jogo, carregar um jogo salvo, ou terminar o aplicativo.}, o
\emph{mapa da cidade}\footnote{Mapa por meio do qual o jogador obtém
  acesso à ações como roubar um estabelecimento, fazer compras no
  mercado, etc. (vide o documento de game design, na
  seção~\ref{fluxo-jogo}).}, \emph{diálogos} e \emph{cenas}, 
\emph{gerência de seu inventário de itens}, \emph{compra no mercado} e
\emph{planejamento do roubo}\footnote{Sequência de telas que
  compreende a seleção de participantes do roubo, equipamento a usar,
  e disribuição das recompensas.}. Cada estado é uma abstração de um
conjunto de interações que o jogador pode ter com o jogo.

Foram encapsuladas as primitivas da biblioteca SDL em classes que
\emph{gestoras de recursos}. Esses recursos são
\begin{itemize}
\item imagens (carregadas para serem exibidas na tela,
\item sons,
\item fontes, e
\item eventos.
\end{itemize}

A conexão com o subsistema Java também foi feita encapsulando-se uma
interface SDL, e adaptou-se o sistema interno de captação e
comunicação de eventos do SDL para incorporar as mensagens do
protocolo de comunicação elaborado (vide apêndice~\ref{ap:protocolo}).

Outras abstrações implementadas incluem \emph{design-patterns} como
\emph{decoration}, \emph{strawman}, \emph{singleton}, \emph{factory} e
\emph{interface}. Por exemplo, objetos que queiram ser informados de
eventos do protocolo devem ser \emph{EventListeners} e se inscrever na
lista de ouvintes do \emph{EventManager}, servidor de mensagens do
jogo. Analogamente, objetos que queiram ser desenhados devem ser
\emph{Drawable} e se inscreverem (ou serem inscritos) em alguma
\emph{View} (objetos gerenciadores de exibição de imagens na
tela).

Finalmente, mas não menos importante, o laço principal do jogo
constitui-se da constante medida do tempo decorrido (em milisegundos)
desde a última execução do laço. Este valor é passado para um método
de atualização do gerenciador de estados
(\emph{StateManager}\footnote{Os estados são objetos em uma pilha,
  gerenciada pelo \emph{StateManager}. Ele também é o responsável pela
criação e destruição dos estados.}), que faz o controle da taxa
de atualização da tela.




\section{Protocolo de comunicação}

A concepção do protocolo de comunicação envolveu alguns cuidados especiais, e precauções foram tomadas no sentido de preservar a flexibilidade do protocolo --- isto é, facilitar a adição de novas mensagens no decorrer do projeto --- e garantir certo desacoplamento entre o trabalho de modelagem de agentes e projeto do blackboard, de um lado, e a lógica do jogo, de outro. Buscou-se organizar a comunicação entre Java e C++ de modo tal que o projeto pudesse ser levado a cabo sem a necessidade de programadores de um e outro sistema precisassem saber do funcionamento do outro. Essa flexibilidade permite que as responsabilidades dos sistemas sejam alteradas durante a evolução do projeto, o que é importante para permitir a exploração de possibilidades de organização do jogo como um todo. Para isso, é necessário que os protótipos construídos não enfrentem as restrições de uma arquitetura rígida.

\section{Estrutura de pastas}\label{estruturaPastas}

O projeto foi desenvolvido com vistas a se prestar à experimentação no
design. Assim, algumas ferramentas para a escrita de \emph{scripts} de
diálogos ou pequenas cenas foram desenvolvidas.

Descrevemos a seguir a organização dos arquivos do projeto. É
importante que essa estrutura seja clara e intuitiva, já que é nossa
intenção que alguns desses arquivos (que configuram o comportamento
do jogo) sejam alterados primariamente por designers do jogo. Assim, é
preciso que haja critério na complexidade que se expõe a esses
``usuários''. A figura~\ref{fig:estrut-arquiv} exibe a distribuição de arquivos nas pastas do projeto.  

\begin{figure}
\centering
\includegraphics[width=\textwidth]{figuras/estrutura-arquivos.jpg}
\caption{Organização de arquivos do projeto}
\label{fig:estrut-arquiv}
\end{figure}


\section{Arquivos auxiliares}

Uma série de arquivos de texto auxiliares são usados pelo
jogo. São em sua maioria estáticos, e sendo que alguns são lidos por ambos os programas. Ressalta-se que os conjuntos de arquivos gerados por Java e C++ não têm intersecção, o que elimina a necessidade de gerência de situações de escrita concorrente. Eles especificam
\begin{itemize}
\item diálogos,
\item modelos de agentes,
\item informações de instâncias de agentes,
\item tipos de reações,
\item tipos de estímulos a agentes,
\item lugares passíveis de assalto,
\item dados de capangas,
\item ações da polícia.
\end{itemize}

Explicamos a seguir a função de cada um dos tipos de arquivo.

Os \emph{diálogos} são scripts que codificam possíveis dizeres que
\npc{}s ou o jogador podem efetuar em um diálogo. Carregam informação sobre
o sentimento que expressam ou a impressão que causam. Seguem a
gramática descrita no apêndice~\ref{ap:gram-script-dialogo}.

Os \emph{modelos de agentes} são arquivos de texto para controle das informações de cada agente. Possuem as seguintes informações sobre os agentes:
\begin{itemize}
\item tipo do agente: inteiro que define o tipo do agente (0-capanga,1-civil,2-policial,3-oculto).
\item id do agente: inteiro único, entre 10 e 99 que representa o agente.
\item nome do agente: String com o nome do agente.
\item personalidade: String que define a personalidade atual do agente.
\item estado psicologico: String que define o estado psicológico atual do agente.
\end{itemize}
Os itens listados acima são comuns a todos os tipos de agentes.
Além dos itens listados acima, os arquivos de agentes policiais possuem uma informação extra sobre a variável \emph{conduta} associada a cada policial, que é uma String que pode assumir dois valores (ganancioso ou íntegro), e os arquivos de agentes capangas possui a variável \emph{nível de suspeita} que é um inteiro (o valor máximo desta variável é definido no arquivo de configuração), ela serve para a polícia monitorar os capangas e decidir se deve ou não prendê-lo.
Vale citar que existe um arquivo por agente e que o nome do arquivo é sempre formado pela seguinte regra: \verb!a! + \verb!tipo-do-agente! + \verb!id-do-agente.txt!

Os \emph{tipos de estímulos, tipos de reações, lugares passíveis de assalto, dados de capangas,} e \emph{ações da polícia} são arquivos de interface, lidos por ambos os programas na decodificação de mensagens trocadas. Optou-se por essa solução para que a comunicação se desse pelo envio de inteiros, indicando qual a linha do arquivo de interface
em que a mensagem está contida. Com isso o processo de alteração do protocolo em seus estágios iniciais de desenvolvimento foi simplificado, uma vez que sua própria especificação atua como um tipo de documentação; além disso, a mudança das mensagens do protocolo passou a ser feita pela mudança de algumas linhas em um arquivo de configuração e em um mapa de strings a funções, o que facilitou a experimentação. Claramente, essa abordagem só é
possível porque as mensagens são conhecidas \emph{a priori}.

Para comunicar o estímulo que uma determinada fala no diálogo provoca
no agente, a lógica do jogo envia ao sistema BDI um inteiro indicando
a linha em que está escrita a string que define o estado. Na prática,
a lógica do jogo faz uma chamada para enviar o inteiro correspondente
ao estímulo que se deseja comunicar, e uma busca é feita no arquivo
que contém os estímulos possíveis para encontrar o conteúdo da linha
correspondente. Essa string é, por fim, enviada ao sistema BDI, que
fará a sua própria busca para identificar qual o estímulo recebido.

Vale notar que os arquivos contendo os estímulos e reações são gerados por um programa que os extrai dos diálogos escritos para o jogo.



