\chapter{Implementação}

\section{Visão global}

Como as coisas funcionam. O que faz o quê (macroscopicamente).

Uma explicação sucinta da estrutura do programa, como ele inicia
(forking) e como age em regime (responsabilidades java vs C++, e o
protocolo).

\section{Compromissos (\emph{trade-offs})}\label{sec:tradeofss}

\section{Arquitetura}

\section{Protocolo de comunicação}

\section{Estrutura de pastas}

O projeto foi desenvolvido com vistas a se prestar à experimentação no
design. Assim, algumas ferramentas para a escrita de \emph{scripts} de
diálogos ou pequenas cenas foram desenvolvidas. 

Descrevemos a seguir a organização dos arquivos do projeto. É
importante que essa estrutura seja clara e intuitiva, já que é nossa
intenção que alguns desses arquivos (que configuram o comportamento
do jogo) sejam alterados primariamente por designers do jogo. Assim, é
preciso que haja critério na complexidade que se expõe a esses
``usuários''.


\section{Arquivos auxiliares}

Uma série de arquivos de texto auxiliares são usados pelo
jogo. Eles especificam
\begin{itemize}
\item diálogos,
\item agentes,
\item estados de agentes,
\item tipos de estímulos a agentes,
\end{itemize}

Explicamos a seguir a função de cada um dos tipos de arquivo.

Os \emph{diálogos} são scripts que codificam possíveis dizeres que
\npc{}s ou o jogador podem optar por dizer. Carregam informação sobre
o sentimento que expressam ou a impressão que causam. Seguem a
gramática descrita no apêndice~\ref{ap:gram-script-dialogo}.

\emph{Agentes} são especificados em \emph{AgentSpeak}, a linguagem
interpretada pela ferramenta \emph{Jason}. Cada arquivo contém a
modelagem de um tipo de agente.

Os três últimos itens são arquivos de interface, usados para
leitura pelos processos de renderização e de interpretação do sistema
BDI. Optou-se por essa solução porque a comunicação se dá pelo envio de
números, que, nesse caso, indicam qual a linha do arquivo de interface
em que a mensagem está contida. Obviamente, essa abordagem só é
possível porque as mensagens são conhecidas \emph{a priori}.

Para comunicar o estímulo que uma determinada fala no diálogo provoca
no agente, a lógica do jogo envia ao sistema BDI um inteiro indicando
a linha em que está escrita a string que define o estado. Na prática,
a lógica do jogo faz uma chamada para enviar o inteiro correspondente
ao estímulo que se deseja comunicar, e uma busca é feita no arquivo
que contém os estímulos possíveis para encontrar o número da linha
correspondente. Esse número é, por fim, enviado ao sistema BDI, que
fará a sua própria busca para identificar qual o estímulo recebido.

No caso de funções, o arquivo de interface deve ser gerado
automaticamente pelo código Java, para que seja consultado durante a
execução pela lógica do jogo.

