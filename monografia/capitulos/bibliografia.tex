\begin{thebibliography}{99}
 \bibitem{cpqd_modelo_ref}
        BRASIL. {C}entro de {P}esquisas e {D}esenvolvimento em {T}elecomunicações ({CPqD}).
        \textbf{MODELO DE REFERÊNCIA}: Sistema Brasileiro de Televisão Digital Terrestre.
        Versão PD.30.12.36A.0002A/RT-08-AB.
        [Campinas]: CPqD, 13 fev. 2006.
        (Relatório Técnico, Cliente: FUNTTEL, OS: 40539)
        Disponível em <\url{http://sbtvd.cpqd.com.br/}> (seção de Divulgação).
        Acesso em 12 ago. 2007.
 
\bibitem{tese_roberto}
        BIANCHINI, R. C. \textbf{Uma arquitetura BDI para comportamentos interativos de agentes em jogos computacionais}. Tese de doutoramento --- Escola Politécnica da Universidade de São Paulo, São Paulo, 2005.
 
\bibitem{livro_russel}
        NORVIG, P.; RUSSEL, S. \textbf{Inteligência artificial}. 1 ed. São Paulo: Elsevier editora ltda, 2007. ISBN: 978-85-352-2564-8.
 
\bibitem{brian_schwab}
SCHWAB, B. \textbf{AI game engine programming}. 2 ed. Hingham: Charles River Media, 2008.
 
\bibitem{wooldrige_agentes}%[4]
WOOLDRIDGE, M. \textbf{Reasoning about rational agents}. Cambridge: The M. I. T. Presss, 2000.
 
\bibitem{oliveira_BDI}%[5]
 
NUNES, I. O. \textbf{Implementação do modelo e da arquitetura BDI}. Monografias em Ciência da Computação --- Pontifícia Universidade Católica do Rio de Janeiro, Rio de Janeiro, 2007.
ISSN 0103-9741
 
\bibitem{design_games}
SCHUYTEMA, P. \textbf{Design de Games: uma abordagem prática}. São Paulo: Editora Cengage, 2008
 
\bibitem{artigo_turing}
TURING, A.M. \textbf{Computing machinery and intelligence}. Mind, 1950. Disponível em: <http://www.loebner.net/Prizef/TuringArticle.html>. Acesso em: 20 ago. 2010.
 
\bibitem{beejs}
HALL, B. \textbf{Beej’s guide to network programming}: using internet sockets. 2009.Disponível em: <http://beej.us/guide/bgnet/output/html/multipage/index.html>.
Acesso em: 30 jul. 2010.
 
\bibitem{deitel}
DEITEL, H.M.; DEITEL, P. J.\textbf{Java}:como programar. 6 ed. São Paulo: Pearson Prentice Hall, 2005. ISBN: 85-7605-019-6.
 
\bibitem{AndrewDave-arquitet-e-design}
ROLLINGS, A.; MORRIS, D. \textbf{Game Architecture and Design}. Albany: Coriolis Group Books Editora, 1999. ISBN 0-7357-1363-4.
 
\bibitem{John-Omar-SwEngForGames}
        FLYNT, J.; SALEM, O. \textbf{Software engineering for game developers}. Boston, MA: Course Technology PTR Editora, 2005. ISBN: 1-59200-155-6.
 
\bibitem{Steve-AIgameWisdom}
        RABIN, S. \textbf{AI game programming wisdom}. Hingham, Mass.: Charles River Media Editora, 2002. ISBN: 978-1584500773.
 
 
\bibitem{design-patterns}
GAMMA, E.; et al. \textbf{Design patterns : elements of reusable object-oriented software} Reading, Mass. : Addison-Wesley Editora, 1995. ISBN: 978-0201633610
 
\bibitem{thining-cpp}
        ECKEL, B. \textbf{Thinking in C++}. Upper Saddle River (N. J.) : Prentice Hall Editora. 2000. ISBN: 978-0139798092.
 
\bibitem{accel-cpp}
        KOENIG, A.; MOO, B. E. \textbf{Accelerated C++ : practical programming by example}. Boston, MA : Addison-Wesley Editora, 2000. ISBN: 978-0201703535.

\bibitem{Julien-oogd}
        GOLD, J. \textbf{Object-oriented game development}.Harlow, England ; New York : Addison Wesley Editora, 2004. ISBN: 978-0321176608
 
\end{thebibliography}
