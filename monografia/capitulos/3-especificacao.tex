\chapter{Especificação}

Nesta seção descrevemos os critérios de aceitação do projeto:
\begin{itemize}
\item descrição das telas,
\item opções esperadas dos menus,
\item comportamento esperado dos \npc{}s
\end{itemize}

O Game Design, que é um dos anexos da monografia, complementa a
especificação do projeto.


\section{Decisões de projeto}

\subsection{Limitações e controle da expressão dos agentes}

Apesar de variabilidade e adaptabilidade da experiência do jogo
constituírem, em grande medida, o que se busca ao introduzir-se algum
modelo de inteligência na lógica de jogos, é preciso ter em mente que
isso não significa que se queira abrir mão do design dessa
experiência. De fato, parte da complexidade da incorporação de
inteligência artificial a jogos advém da necessidade de identificar
que aspectos da dinâmica da inteligência devem ser coergidos, moldados
ou mesmo eliminados, em prol de atributos como a jogabilidade e mesmo
adaptação às limitações e anseios dos jogadores.

Nesse sentido, optou-se por um modelo --- experimental e
por vezes potencialmente limitante, como será visto adiante ---
restritivo de expressão dos agentes que controlam o diálogo com
\npc{}s.

O modelo implementado tem por ideia central que os diálogos são
interações de \emph{estímulo e reação} do agente inteligente pelo
jogador. Os estímulos estão associados ao que o jogador opta por
dizer, e as reações às opções de fala do \npc. 

As opções de um e outro estão registradas nos \emph{roteiros de
  diálogo}, que são compostos pelos designers do jogo. Nesses
roteiros, as falas de jogador e agente estão descritas, e tanto os
estímulos que transmitem quanto as reações que expressam são definidos
por uma \emph{marcação de conteúdo}. Assim, uma saudação pode ser um
estímulo amigável; uma afirmação pode soar suspeita; e, por outro
lado, uma interjeição pode expressar contentamento ou rabugice.

Desse modo, o conjunto de condições passíveis de
expressão pelos agentes fica sob controle do designer do jogo, por
causa da estreita ligação entre os diálogos e os estímulos e
manifestações disponíveis para os agentes. Essa é uma característica
extremamente desejável, como já foi dito, por permitir introduzir a
variabilidade de maneira controlada e em sintonia com a experiência
que se quer produzir.

Existem algumas dificuldades inerentes a esse processo de
modelagem. Está sujeito a avaliação se em projetos de maior porte a
tarefa de gerência dos vários estímulos e reações não se tornaria
proibitiva do uso da técnica.


\section{Compromissos (\emph{trade-offs})}
