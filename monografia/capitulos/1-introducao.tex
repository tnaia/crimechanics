\chapter{Introdução}


%Motivação e referências básicas do trabalho.
\section{Justificativas do trabalho}
O uso de técnicas de inteligência artificial em jogos eletrônicos vem sendo feito desde a década de 70. Desde então, as mais diversas técnicas já foram utilizadas no desenvolvimento de jogos,  entretanto, há ainda um grande número de desafios na abordagem de problemas como a percepção de contextos em jogos\footnote{Em particular pois trata-se de contextos intrincados, como tramas de histórias em jogos de RPG.} --- campo de aplicabilidade do blackboard ---, e também da adaptabilidade e imprevisibilidade de resposta e interação com (e entre) \npc{}s, assunto onde proliferam modelos de abstracão do raciocínio humano, como o BDI (Belief-Desire-Intention).

\section{Objetivos}
O principal objetivo deste projeto é estudar os conceitos e aplicação da técnica Blackboard de inteligência artificial em um jogo computacional, juntamente com a arquitetura BDI (Belief-Desire-Intentions) de agentes inteligentes. 

Além disso, o grupo vivenciará a experiência de desenvolver um jogo computacional.
Para isso, foram realizados estudos sobre o processo desenvolvimento de jogos, os conceitos e tecnologias utilizadas.
