This work discusses the development of a case study of the use of artificial intelligence techniques to a game. The study applied both the \emph{Belief-Desire-Intention} (BDI) architecture for cognitive agents and blackboard technique (for knowledge sharing among agents) in order to enrich the complexity and variability of reaction of the game and its \emph{non-palyer characters} (\npc{}s). Aspects of the efficiency of the use of an interpreter for an agent-oriented programming language, alongside with a compiled rendering game logic control system were considered.

The work allows us to observe the flexibility gain that such techniques potentially add to the game design, as well as technical limitations arising from the growth of modeling complexity, as choices and contexts dealt with by the agents multiply.

It follows that even small games can benefit from the use of \emph{artificial intelligence} to interactions between the player and \npc{}s.

Keywords: Engineering. Computer Engineering. Artificial
Intelligence. BDI Architecture. Blackboard System. Games.
