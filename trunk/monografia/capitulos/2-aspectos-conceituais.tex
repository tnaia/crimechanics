\chapter{Aspectos Conceituais}

\section{Inteligência Artificial}

%Um pouco de história, referências básicas sobre o assunto, usos mais
%comuns, estado da arte.

A Inteligência Artificial (IA) é uma área de pesquisa da ciência da computação e Engenharia da Computação, dedicada a buscar métodos ou dispositivos computacionais que possuam ou simulem a capacidade racional de resolver problemas, pensar ou, de forma ampla, ser inteligente.

Apenas recentemente, com o surgimento do computador moderno, é que a inteligência artificial ganhou meios e massa crítica para se estabelecer como ciência integral, com problemáticas e metodologias próprias. Desde então, seu desenvolvimento tem extrapolado os clássicos programas de xadrez ou de conversão e envolvido áreas como visão computacional, análise e síntese da voz, lógica difusa, redes neurais artificiais e muitas outras.
Inicialmente a IA visava reproduzir o pensamento humano. A Inteligência Artificial abraçou a idéia de reproduzir faculdades humanas como criatividade, auto-aperfeiçoamento e uso da linguagem. Porém, o conceito de inteligência artificial é bastante difícil de se definir. Por essa razão, Inteligência Artificial foi (e continua sendo) uma noção que dispõe de múltiplas interpretações, não raro conflitantes ou circulares.

\subsection{História}

O primeiro trabalho a respeito de IA foi realizado por Warren McCulloch e Walter Pitts em 1943, em que demonstrava um modelo de neurônios artificiais, em que cada neurônio poderia estar ``ligado'' ou ``desligado'', isso dependeria dos estados em que os neurônios vizinhos estariam. (RUSSEL \& NORVIG, 2004: 18)

Donald Hebb demonstrou em 1949 uma regra de atualização simples para definir a intensidade de conexão entre neurônios, essa regra é influente ainda nos dias de hoje (RUSSEL \& NORVIG, 2004: 18).

Em 1950 Alan Turing publicou o artigo \textit{``Computing Machinery and Intelligency''}. Neste artigo ele apresentou diversos algoritmos de IA e seu famoso teste, o ``Teste de Turing''. (Idem, 18).

Em 1951, no departamento de matemática de Princeton, Marvin Minsk e Dean Edmound apresentaram o primeiro computador de rede
neural, o SNARC era composto por 3000 válvulas e simulava uma rede de 40 neurônios. (Op. Cit.,18)

Segundo RUSSEL \& NORVIG (2004: 18-19), cinco anos após o SNARC John McCarthy realizaram um seminário em Dartmouth, reunindo todos os grandes pesquisadores do conhecimento. Nesse seminário Allen Newell e Herbert Simon do Carnegie Tech apresentaram o \textit{Logic Theorist} o primeiro programa de raciocínio. Nesse seminário John McCarthy juntamente com os outros seminaristas percebeu que a IA devia ser um campo em separado e não uma ramificação da matemática nem mesmo estar sob o nome de pesquisa operacional ou teoria de controle, então eles definiram o nome de Inteligência Artificial e, após este seminário a IA foi definida como o campo em que tenta construir máquinas que funcionem de forma autônoma e em ambientes mutáveis.


Pesquisas sobre inteligência artificial foram intensamente custeadas na década de 1980 pela Agência de Projetos de Pesquisas Avançadas sobre Defesa (\textit{“Defense Advanced Research Projects Agency”}), nos Estados Unidos, e pelo Projeto da Quinta Geração (\textit{“Fifth Generation Project”}), no Japão. O trabalho subsidiado fracassou no sentido de produzir resultados imediatos, a despeito das promessas grandiosas de alguns praticantes de IA, o que levou proporcionalmente a grandes cortes de verbas de agências governamentais no final dos anos 80, e em conseqüência a um arrefecimento da atividade no setor, fase conhecida como O inverno da IA. No decorrer da década seguinte, muitos pesquisadores de IA mudaram para áreas relacionadas com metas mais modestas, tais como aprendizado de máquinas, robótica e visão computacional, muito embora pesquisas sobre IA pura continuaram em níveis reduzidos.

Os fatos acima listados marcam o nascimento e desenvolvimento da inteligência artificial, daquela época até os dias de hoje ocorreram grandes avanços neste ramo da computaçao. 

Atualmente a IA está mais madura e não possui mais aquele ar de pioneirismo do inicio. O pesquisador de IA deve provar suas teorias com diversos experimentos empíricos e a demonstração dos resultados, e com o surgimento da Internet os pesquisadores foram capazes de compartilhar as evoluções de suas pesquisas e acelerar os processos de desenvolvimento.

Encorajados pela resolução dos subproblemas da IA, os pesquisadores começaram a examinar o problema do ``agente como um todo''. O movimento estabelecido tem como objetivo entender o funcionamento interno de agentes incorporados a ambientes reais com entradas sensoriais continuas. Um dos ambientes mais importantes para os agentes inteligentes é a Internet, diversas ferramentas utilizam a Inteligência Artificial na Internet como mecanismo de pesquisa, sistemas de recomendação e sistemas de construção de web sites. (RUSSEL \& NORVIG, 2004:27-28)
Na tentativa de se construir agentes percebeu-se que subcampos que anteriormente eram isolados da IA, precisavam de uma reorganização para o melhor aproveitamento de seus resultados.
Outra conseqüência foi que a IA se aproximou de outras áreas, como a teoria de controle e a economia, áreas que também lidam com agentes. (RUSSEL \& NORVIG, 2004:28)
No século XXI há um grande interesse em \textit{Agentes Inteligentes} uma nova ramificação que pretende reunir todas as subáreas da IA, pois se percebeu que todas juntas poderiam criar um Agente perfeito realizando o principal objetivo da IA, entender o pensamento humano.

Entre os teóricos que estudam o que é possível fazer com a IA existe uma discussão onde se consideram duas propostas básicas: uma conhecida como ``forte'' e outra conhecida como ``fraca''.
A investigação em Inteligência Artificial Forte aborda a criação da forma de inteligência baseada em computador que consiga raciocinar e resolver problemas uma, sendo assim a forma de IA forte é classificada como autoconsciente.
A Inteligência Artificial Fraca centra a sua investigação na criação de inteligência artificial que não é capaz de verdadeiramente raciocinar e resolver problemas. Uma tal máquina com esta característica de inteligência agiria como se fosse inteligente, mas não tem autoconsciência ou noção de si.


\subsection{Áreas de aplicação}
Enquanto que o progresso direcionado ao objetivo final de uma inteligência similar à humana tem sido lento, muitas derivações surgiram no processo. Exemplos notáveis incluem as linguagens LISP e Prolog, as quais foram desenvolvidas para pesquisa em IA, mas agora possuem funções não-IA. A cultura Hacker surgiu primeiramente em laboratórios de IA, em particular no MIT AI Lab, lar várias vezes de celebridades tais como McCarthy, Minsky, Seymour Papert (que desenvolveu a linguagem Logo), Terry Winograd (que abandonou IA depois de desenvolver SHRDLU).
Muitos outros sistemas úteis têm sido construídos usando tecnologias que ao menos uma vez eram áreas ativas em pesquisa de IA. Alguns exemplos incluem:
\begin{itemize}
\item Planejamento automatizado e escalonamento: a uma centena de milhões de quilômetros da Terra, o programa Remote Agent da NASA se tornou o primeiro programa de planejamento automatizado (autônomo) de bordo a controlar o escalonamento de operações de uma nave espacial.
\item Jogos: O Deep Blue da IBM se tornou o primeiro programa de computador a derrotar o campeão mundial em uma partida de xadrez, ao vencer Garry Kasparov.
\item Controle autônomo: O sistema de visão de computador ALVINN foi treinado para dirigir um automóvel, mantendo-o na pista.
\item Robótica: Muitos cirurgiões agora utilizam robôs assistentes em microcirurgias.
\item Lógica incerta: Técnica para raciocinar dentro de incertezas, tem sido amplamento usada em sistemas de controles industriais.
\item Redes Neurais: Vêm sendo usadas em uma larga variedade de tarefas, desde sistemas de detecção de intrusos a jogos de computadores.
\item Aplicações utilizando Vida Artificial são utilizados na indústria de entretenimento e no desenvolvimento da Computação Gráfica.
\item Sistemas baseados na idéia de agentes artificiais, denominados Sistemas Multiagentes, têm se tornado comuns para a resolução de problemas complexos.
\end{itemize}
Atualmente existem muitas aplicações práticas que envolvem conceitos de inteligência artificial. Dentre todas as áreas que, de alguma forma implementam conceitos de IA, os jogos eletrônicos tem ganhado grande destaque nos últimos anos.

\section{Jogos digitais}

Desde a criação do primeiro jogo digital até os dias de hoje ocorreram muitas evoluções. Atualmente os jogos possuem gráficos que beiram a realidade.
\textbf{Falar sobre a evolução gráfica nos jogos! Os gráficos estão tão avançados que só falta a lógica ser legal!}
\textbf{A mim, os jogos que mais me istigram foram os de aventura/RPG como zelda e Vampire onde vc controla um personagem em um mundo e interage com vários NPCs mas sempre com as mesmas falas e ações, ou seja, não tem dinâmica.}
%Uma curta evolução histórica, tendências atuais (puxando um pouco para os problemas de IA --- a questão da imprevisibilidade e adaptabilidade dos jogos). Referências.

\section{Inteligência em jogos}

O principal objetivo do uso de técnicas de inteligência artificial em jogos eletrônicos é a diversão. Por causa disso, o conceito de IA acabou recebendo outra interpretação por parte dos desenvolvedores de jogos, surgiu o conceito de \textit{Game AI}.

Diferentemente da IA acadêmica que busca solucionar problemas extramamente difíceis, como imitar o reconhecimento que os humanos são capazes de realizar (reconhecimento facial e de imagens e objetos, por exemplo), ou mesmo entender e construir agentes inteligentes, a IA para jogos se preocupa com os resultados que o sistema irá gerar, e não como o sistema chega até os resultados. Isso se deve ao fato que jogos eletrônicos são negócios e os consumidores desses produtos os compram em busca de diversão, e não lhes interessa como a inteligência de um personagem no jogo foi criada, desde que ela transforme o jogo divertido e desafiador, além, claro, de tomar decisões coerentes com o contexto do jogo.

Um exemplo que deixa claro a utilidade da IA para jogos são os \textit{shooters}. Nos jogos de tiros existem várias técnicas de IA que podem ser aplicadas, por exemplo, é possível fazer com que os \textit{Non-Player Characters} (NPCs) se comuniquem e criem estratégias para tentar cercar um inimigo, fato que pode tornar o jogo ainda mais interessante, no entanto, também é possível utilizar técnicas de IA para fazer com que os NPCs acertem todos os disparos na cabeça de seus inimigos, fato que passa longe da realidade humana, e que poderia prejudicar a qualidade do jogo do ponto de vista de um usuário.

No começo do desenvolvimento de jogos eletrônicos, a programação de IA era mais usualmente conhecida por \textit{programação de jogabilidade}, pois não havia nada de inteligente sobre os comportamentos exibidos pelos personagens controlados pelo computador. 
A figura abaixo \textbf{(figura IA jogos (SCHWAB, 2004) SCHWAB, Brian. AI Game Engine Programming. Hingham: Charles River Media. 2004.)}contém alguns exemplos de como a IA foi utilizada em jogos com o passar do tempo.

%Um pouco de história, motivação do uso, desafios e problemas comuns.

\subsection{Crença, desejo e intenção --- a arquitetura BDI}

O modelo BDI (\textit{Beliefs Desires Intentions}) foi originalmente proposto por Bratman como uma teoria filosófica do raciocínio prático, propondo uma análise do comportamento humano que seria baseado em crenças, desejos e intenções.
Basicamente supõe-se que as ações são derivadas a partir de um processo chamado raciocínio prático. Este processo é constituído por duas etapas, na primeira, deliberação, o agente seleciona um conjunto de desejos que devem ser alcançados, de acordo com a situação atual das crenças do mesmo. Na segunda etapa ocorre a determinação de como os desejos produzidos no passo anterior podem ser atingidos através do uso dos meios disponíveis ao agente \textbf{[4] Michael Wooldridge, Reasoning about Rational Agents. Cambridge, MA: The M. I. T. Presss 2000.}.
A seguir explicaremos melhor o que são crenças, desejos e intenções.
\begin{itemize}
\item \textbf{Crenças (Beliefs)}: Representam as características do ambiente e são atualizadas constantemente. Podem ser vistas como a componente informativa do sistema.
\item \textbf{Desejos (Desires)}: Representam os objetivos a serem alcançados. Podem ser vistos como motivações do sistema.
\item \textbf{Intenções (Intentions)}: Representam o atual plano de ações escolhido. 
\end{itemize}

A partir deste modelo nasceu a arquitetura BDI para agentes. Agentes BDI são sistemas localizados em um ambiente (em nosso caso o ambiente será virtual) sujeito a variações, além disso, percebem o estado deste ambiente constantemente e podem atuar sobre o mesmo para tentar alterar o estado atual.
Abaixo vemos o processo de raciocínio prático de um agente BDI.
 
\textbf{Figura 2 - Diagrama de uma arquitetura BDI genérica [4]}

Como se pode observar existe sete elementos básicos que compõem um agente BDI, são eles \textbf{[5] Ingrid Oliveira de Nunes, Implementação do modelo e da arquitetura BDI}:
\begin{itemize}
\item Um conjunto de crenças (\textit{Desires}) atuais que representam as informações que o agente tem do ambiente.
\item Uma função de revisão de crenças (\textit{Belief Revision Function}), a qual determina um novo conjunto de crenças a partir da percepção da entrada e das crenças do agente.
\item Uma função de geração de opções (\textit{Option Generation Function}), a qual determina as opções disponíveis ao agente (seus desejos), com base nas suas crenças sobre seu ambiente e nas suas intenções.
\item Um conjunto de opções (\textit{desires}) corrente que representa os possíveis planos de ações disponíveis ao agente.
\item Uma função de filtro (\textit{filter}), a qual representa o processo de deliberação do agente, que determina as intenções do agente com base nas suas crenças, desejos e intenções atuais.
\item Um conjunto de intenções (\textit{Intentions}) atual, que representa o foco atual do agente, isto é, aqueles estados que o agente está determinado a alcançar.
\item Uma função de seleção de ação (\textit{Action Selection Function}), a qual determina uma ação a ser executada com base nas suas intenções atuais. 
\end{itemize}

\subsection{Quadro-negro --- blackboard}

	A forma mais simples de apresentar o conceito de blackboard é através de uma metáfora que propõem a seguinte situação \textbf{(Daniel D. Corkill, Blackboard Systems. Blackboard Technology Group, Inc.)}:
 “Imagine um grupo de cientistas reunidos em uma sala trabalhando de forma cooperativa para resolver um problema. Para chegar a uma solução é utilizado um quadro negro (\textit{blackboard}).
A resolução do problema começa quando o mesmo é escrito no quadro negro juntamente com informações iniciais. Os cientistas verificam o conteúdo do quadro e aguardam uma oportunidade para aplicar seus conhecimentos visando solucionar o problema. Quando um cientista encontra informações suficientes para fazer uma contribuição, o mesmo coloca sua contribuição no quadro negro, e com sorte, este processo ativará outro cientista e assim por diante até chegarem à solução do problema.”
	Da metáfora acima se pode concluir que um blackboard possui uma base de dados comum a diversos sistemas que estão tentando solucionar um problema e utilizam e atualizam as informações existentes nesta base de dados.
	 A seguir apresentaremos algumas características desta técnica:
	 \begin{itemize}
	 \item \textbf{Independência de conhecimento}: No exemplo acima, supomos que cada cientista adquiriu seus conhecimentos de maneira independente dos outros, ou seja, num blackboard cada sistema que participa da solução de um problema tem seus próprios níveis de conhecimento, independentemente dos outros sistemas.
	 \item \textbf{Diversidade nas técnicas de solução de problemas}: Uma das grandes vantagens do blackboard é que não interessa como os sistemas que estão trabalhando na resolução do problema funcionam, ou seja, um sistema pode utilizar redes neurais enquanto outro pode utilizar simulações, que para o blackboard estas “fontes de conhecimento” são caixas pretas que fazem suas contribuições.
	 \item \textbf{Representação da informação é flexível}: Não existe nenhuma definição de como deve ser a informação, dando liberdade a quem implementa de definir como representar os dados.
	 \item \textbf{Linguagem comum entre os sistemas}: Ao mesmo tempo em que existe a flexibilidade na representação da informação, é necessário, por outro lado, que todos os sistemas envolvidos sejam capazes de entender tais informações.
	 \item \textbf{Liberdade de organização dos dados}: A organização dos dados é livre, porém, deve ser feita de tal forma que, no caso da base de dados possuir muita informação, seja fácil para um sistema encontrar informações específicas de forma rápida e simples.
	 \item \textbf{Ativações baseadas em eventos}: Os sistemas que estão trabalhando na solução do problema não se comunicam diretamente, ao invés disto, todos “observam” a base de dados comum e utilizando as informações da mesma buscam gerar novos dados que são colocados na mesma base, tais dados podem ativar outro sistema que fará a mesma coisa até que o problema seja completamente resolvido.
	 \item \textbf{Necessidade de controle}: É preciso haver um “órgão” para controlar todos os sistemas e decidir quem deve e quem não deve poder alterar os dados do quadro negro, assim não existe o risco de mais de um sistema alterar uma mesma área de dados simultaneamente.
	 \item \textbf{Geração de solução incremental}: É evidente que a solução de um problema acontece de passo em passo, ou seja, um sistema faz uma contribuição com novos dados, a seguir, outro sistema utiliza tais dados e faz uma nova contribuição e de ciclo em ciclo o problema tende a ser resolvido. 
	 \end{itemize}

Abaixo temos uma visão geral de como funciona o blackboard. Basicamente existem fontes de conhecimento (sistemas que estão trabalhando na resolução do problema) um sistema de controle que concede permissões às fontes de conhecimento para alterarem os dados do quadro negro, este por sua vez, é uma base de dados que contem informações que ficam disponíveis a todas as fontes de conhecimento.

\textbf{Figura 1- Visão geral de um sistema blackboard}
	
Neste projeto utilizaremos a técnica blackboard como parte de um jogo computacional que será desenvolvido.


\subsection{O que avaliar}

O que o estudo de caso vai avaliar, e como. Decisões de projeto que
validam o estudo (C++, \emph{data-driven} design, prototipação,
engenharia de software), isto é, porque o projeto permite extrapolar
conclusões para jogos na indústria.
