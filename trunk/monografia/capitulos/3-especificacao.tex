\chapter{Especificação}

%Nesta seção descrevemos os critérios de aceitação do projeto:
%\begin{itemize}
%\item descrição das telas,
%\item opções esperadas dos menus,
%\item comportamento esperado dos \npc{}s
%\end{itemize}

Neste capítulo abordam-se os requisitos que o projeto deve satisfazer para que o estudo de caso seja efetivo. Foi dada atenção particular à aderência do projeto a padrões industriais de metodologia de desenvolvimento de jogos, na medida em que tal não se mostrasse incompatível com o escopo de um projeto de formatura. 

Assim optou-se pelo emprego do GDD (Game Design Document --- apêndice~\ref{ap:gdd}) como meio de especificação e captura de requisitos do projeto. Da mesma forma foram tomadas decisões como a escolha das linguagens e bibliotecas empregadas, tendo em vista padrões recorrentes na indústria e portabilidade.

\section{Decisões de projeto}

\subsection{Limitações e controle da expressão dos agentes}

Apesar de variabilidade e adaptabilidade da experiência do jogo
constituírem, em grande medida, o que se busca ao introduzir-se algum
modelo de inteligência na lógica de jogos, é preciso ter em mente que
isso não significa que se queira abrir mão do design dessa
experiência. De fato, parte da complexidade da incorporação de
inteligência artificial a jogos advém da necessidade de identificar
que aspectos da dinâmica da inteligência devem ser coergidos, moldados
ou mesmo eliminados, em prol de atributos como a jogabilidade e mesmo
adaptação às limitações e anseios dos jogadores.

Nesse sentido, optou-se por um modelo --- experimental e
por vezes potencialmente limitante, como será visto adiante ---
restritivo de expressão dos agentes que controlam o diálogo com
\npc{}s.

O modelo implementado tem por ideia central que os diálogos são
interações de \emph{estímulo e reação} do agente inteligente pelo
jogador. Os estímulos estão associados ao que o jogador opta por
dizer, e as reações às opções de fala do \npc. 

As opções de um e outro estão registradas nos \emph{roteiros de
  diálogo}, que são compostos pelos designers do jogo. Nesses
roteiros, as falas de jogador e agente estão descritas, e tanto os
estímulos que transmitem quanto as reações que expressam são definidos
por uma \emph{marcação de conteúdo}. Assim, uma saudação pode ser um
estímulo amigável; uma afirmação pode soar suspeita; e, por outro
lado, uma interjeição pode expressar contentamento ou rabugice.

Desse modo, o conjunto de condições passíveis de
expressão pelos agentes fica sob controle do designer do jogo, por
causa da estreita ligação entre os diálogos e os estímulos e
manifestações disponíveis para os agentes. Essa é uma característica
extremamente desejável, como já foi dito, por permitir introduzir a
variabilidade de maneira controlada e em sintonia com a experiência
que se quer produzir.

Existem algumas dificuldades inerentes a esse processo de
modelagem. Está sujeito a avaliação se em projetos de maior porte a
tarefa de gerência dos vários estímulos e reações não se tornaria
proibitiva do uso da técnica.

\subsection{Licença e implicações}

Desde o princípio fez-se a opção por desenvolver um projeto que se adaptasse à filosofia dos softwares livres, de código aberto. Essa característica teve consequências importantes e decisivas no encaminhamento do projeto durante toda a sua evolução. 

Apesar da severidade que as restrições impostas por uma licensa livre ao projeto, deve-se frisar que essa escolha se apresentou certa para o grupo. Não só é inegável o caráter inerentemente livre da produção acadêmica de universidades públicas, meio em que se insere este projeto, como é preciso que se diga que há uma janela de oportunidade na atual carência, no universo dos softwares livres, de projetos de jogos. Essa carência, note-se, vem sendo paulatinamente alvo de estudos e experimentações no mercado, fato que se observa pela progressão do número de títulos que vêm sendo portados para sistemas livres.

Há duas consequências de grande impacto prático na condução do projeto de um jogo livre,  a saber a restrição no uso de software de terceiros e de conteúdo midiático que pode ser aproveitado. Como será discutido adiante (seção~\ref{sec:tradeoffs}), essa opção reduziu a gama de arcabouços de tecnologia que puderam ser aproveitados, e impôs ao projeto a responsabilidade pela criação de parte da arte do jogo --- uma carga de trabalho não desprezível dado escopo e recursos disponíveis.  

\subsection{Tecnologias empregadas}

Uma das primeiras decisões que foram tomadas na definição das condições de contorno do projeto foi que a linguagem empregada para desenvolvimento do software deveria refletir algum padrão solidamente estabelecido na indústria. Optou-se por C++, uma linguagem que goza de grande popularidade na idústria com títulos conhecidos como \emph{Doom} e \emph{Age of Empires}.

Outra decisão de impacto foi a escolha de engines. À partir de uma primeira lista, compilada por meio de pesquisas em fóruns de discussão sobre desenvolvimento de jogos, construiu-se uma matriz de decisão em que foram eliminadas as opções que não apresentavam licensas livres, portáveis e que não possuíssem interface com C++. Por fim escolheu-se empregar o Simple DirectMedia Layer (SDL), linguagem de interface portável com dispositivos de media e periféricos\footnote{Como um adendo, citamos dois títulos conhecidos que usam SDL em pelo menos uma versão: \emph{World of Goo} e \emph{Doom 3}.}.

O uso da ferramenta Jason, para a criação dos agentes, se deve principalmente ao fato de que o projeto nasceu motivado pela tese de doutorado~\cite{tese_roberto} de um dos orientadores do grupo que, no desenvolvimento de seu projeto, utilizou a ferramenta. Além disso, como esta ferramenta é baseada em Java, parte do projeto foi desenvolvida utilizando essa linguagem como se pode observar na figura \ref{algumaCoisaDaArquiteturaDoSoftware!!!}.

%Escolhemos C++ porque é bastante usado na indústria.
%Fizemos uma matriz de decisão cruzando informações sobre diversos engines de %renderização, e baseamos a decisão segundo as restrições: tem que ser livre, tem que ser %portável, tem que dar pra usar com C++.
%Jason -> ferramenta “bonitinha” para se trabalhar com sistemas de agentes inteligentes.
%-> SDL, Jason (plug-in para eclipse), C++.


\section{Especificação dos requisitos do software}
Conforme o modelo de desenvolvimento adotado pela indústria, os requisitos do software foram levantados a partir do GDD e das decisões de projeto. Vale ressaltar que no caso do desenvolvimento de jogos, o documento de consulta primário é o GDD.

É importante citar que a grande maioria dos requisitos funcionais do software são regras do jogo e estão descritos no documento de game design. Dentre os outros requisitos funcionais temos:
\begin{itemize}
\item O software deve fazer e administrar a comunicação entre os módulos implementados em C++ e Java.
\item O diálogo entre o usuário e um \npc{} não pode parecer determinístico.
\end{itemize}

Os requisitos não-funcionais levantados foram os seguintes:
\begin{itemize}
\item A comunicação entre os módulos implementados em Java e C++ não deve demorar mais que 3 segundos.
\item As tecnologias envolvidas no projeto devem ser livres.
\item A interface homem-máquina deve ser simples e intuitiva
\item O tempo de processamento de dados no blackboard não pode ser superior a 5 segundos.
\end{itemize}
