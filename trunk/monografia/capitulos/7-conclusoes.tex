\chapter{Conclusão}

Findo o projeto, resta uma série de considerações a serem feitas,
relativas a diversas dimensões do trabalho desenvolvido. No que
segue, procuramos abordar algumas das questões que foram investigadas,
assim como as que foram surgiram e eventualmente foram deixadas à
margem.


É nossa esperança que esta retomada permita ao leitor vislumbrar um
pouco do fascínio e curiosidade que esses assuntos nos provocaram, e
que, parte delas se espalhe, viceje e reproduza, suscitando sempre
novos questionamentos.

\section{Viabilidade}

No que concerne à aplicação da arquitetura BDI e do blackboard,
chegamos à conclusão de que existem restrições a seu uso. Restrições
em ao menos dois aspectos: conceitual e técnico.

Como discutido na seção~\ref{sec:tradeoffs}, foi observado, com
vantagem para o projeto, o compromisso de limitação da ação
``independente'' do BDI em prol do ganho de controle na composição dos
diálogos. Quanto ao blackboard, existe ainda a questão de encontrar
uma abordagem mais adequada do problema de compartilhamento de
informação. Nosso modelo, simplificado, permitiu avaliar seu
desempenho em quando usado topicamente, o que ainda está longe de
fornecer uma direção de pesquisa em que investigar sua interação com
redes médias e grandes de contextos intersectantes.

De todo modo, observou-se um enriquecimento das possibilidades de
expressão ao se permitir que \npc{}s variem seu comportamento.

Dadas as restrições do projeto, acreditamos que a investigação feita aponta para um resultado promissor, em que técnicas de inteligência artificial permitirão expandir as fronteiras da interação em jogos.

\section{Trabalhos futuros}

Foi identificada uma certa carência de ferramentas de desenvolvimento no decorrer do projeto.
Como a proposta do trabalho foi, de certa forma inovadora, esta carência não foi inesperada. Um caso que se destaca é a falta de ergonomia do método de composição de diálogos que se empregou. Isso pois como tanto o jogador como o \npc{} possuem mais de uma fala possível em alguns nós do diálogo, a árvore de interações possíveis expande rapidamente. Seria interessante desenvolver alguma interface para a  edição desses diálogos, permitindo rápida identificação do locutor, bifurcações e reuniões de linhas do discurso.

Ademais, é interessante que se investigue outros modelos de perfis de agentes, explorando mais as capacidades do AgentSpeak, tanto em variabilidade de perfis como em complexidade dos agentes.

%Seria legal um editor de diálogos, e explorar mais a elaboração de
%perfis de agentes (mais variados e mais complexos)
%Sucesso do estudo de caso?


É importante destacar o papel que a aquisição de conhecimento no
domínio do projeto --- e com isso referimo-nos tanto ao estudo técnico
quanto ao conhecimento adquirido pelo contato com pessoas do ramo ---
na mudança da própria maneira e enfoque do projetar e compreender o
projeto do jogo. Nesse contexto é notável o papel que desempenhou o
curso de uma disciplina do domínio da produção de
jogos\footnote{A referida disciplina é Design e Programação de Games
 (\textsc{PCS2530}), cursada no segundo semestre de 2010.}, assim
como a participação no SBGames, nos quais adiquiriu-se um conhecimento
que, de certo modo, não está acessível por meio das vias consagradas
de estudo; um conhecimento vivo, orgânico, da esfera cultural que é
parte inerente dos jogos neste tempo.

Muito mais do que estávamos cientes, os jogos digitais encontram-se, hoje, em uma posição
privilegiada, na fronteira entre tecnologia e arte, entre ciência e
cultura. Há toda uma dimensão no jogo, uma dimensão além de um
entretenimento que é produto e passa-tempo apenas. Um universo vasto e cheio de possibilidades a explorar.

É nessa perspectiva que o grupo cogita a possibilidade de continuar o desenvolvimento de projetos na linha de investigação da aplicação de técnicas de inteligência artificial a jogos, com vistas a participar do próximo Simpósio Brasileiro de Games, apresentando os resultados dessa pesquisa.
