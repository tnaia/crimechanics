\chapter{Conclusão}

Sucesso do estudo de caso?


É importante destacar o papel que a aquisição de conhecimento no
domínio do projeto --- e com isso referimo-nos tanto ao estudo técnico
quanto ao conhecimento adquirido pelo contato com pessoas do ramo ---
na mudança da própria maneira e enfoque do projetar e compreender o
projeto do jogo. Nesse contexto é notável o papel que desempenhou o
curso de uma disciplina do domínio da produção de
jogos\footnote{A referida disciplina é Design e Programação de Games
  (\textsc{PCS2530}), cursada no segundo semestre de 2010.}, assim
como a participação no SBGames, nos quais adiquiriu-se um conhecimento
que, de certo modo, não está acessível por meio das vias consagradas
de estudo; um conhecimento vivo, orgânico, da esfera cultural que é
parte inerente dos jogos neste tempo.

Jogos encontram-se hoje, na opinião do grupo, em uma posição
privilegiada, na fronteira entre tecnologia e arte, entre ciência e
cultura. E há toda uma dimensão no jogo, uma dimensão além de um
entretenimnto que é produto e passa-tempo apenas, que estava pouco
presente na concepção que anteriormente possuíamos.

\section{Trabalhos futuros}

Próximo SBGames?

\section{\emph{Post Mortem}}

\subsection{\emph{``Lessons learned''}}

