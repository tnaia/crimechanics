\chapter{Metodologia}

O sucesso de um projeto depende em grande medida da capacidade de
adaptação dos envolvidos. Nesta seção tratamos do método de
planejamento de atividades to trabalho, e de sua evolução no decorrer
do projeto. Assim, em certa medida trataremos das mudanças que se
operaram nas expectativas e previsões que fizemos ao longo do tempo.

\section{Evolução}

\begin{enumerate}
\item Levantamento de atividades e assuntos a pesquisar, cronograma
  preliminar (de pesquisas), criação de diário de bordo.%

\item Decisões de projeto iniciais (condições de contorno): ferramentas e
tecnologias de desenvolvimento:
\begin{itemize}
\item  escolha de linguagem de presença expressiva na indústria: C++;
\item  definição de licensa livre para o projeto;
\item  opção por jogo 2D;
\item  pesquisa de simuladores BDI;
\item  identificação de necessidades de renderização;
\item  pesquisa de engines livres.
\end{itemize}%

\item Divisão do projeto em escopos de conhecimento. C++, Java, BDI,
AgentSpeak, SDL, Arquitetura de software para jogos.%

\item Cronograma refinado (com um mês de folga): 
\begin{itemize}
\item escrita do documento de especificação game-design document (GDD);
\item previsão de tempos de implementação, 
  busca de conteúdo artístico, e modelagem;
\item especificação de testes.
\end{itemize}

\item Metodologia de desenvolvimento: controle de versão, práticas de
engenharia de software (referências), prototipação das funcionalidades básicas.
\end{enumerate}

\subsection{Percalços e reajustes}

O que não ocorreu como previsto, e os ajustes que foram feitos. (Falar
em termos mais genéricos, o específico será dito na história de
evolução do projeto.)

Complexidade da modelagem de agentes (?)

