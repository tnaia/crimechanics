Esta monografia aborda o desenvolvimento de um estudo de caso do emprego de técnicas de inteligência artificial a um jogo. Neste estudo foram aplicadas a arquitetura \emph{Belief--Desire-Intention} (BDI) para agentes cognitivos e a técnica blackboard (para compartilhamento de conhecimento entre agentes) de modo a enriquecer a complexidade e variabilidade de reação do jogo e seus \emph{non-player players} (\npc{}s). Foram considerados aspectos da eficiência do uso de um interpretador de linguagem de programação orientada a agentes, em conjunto com um sistema de renderização e controle da lógica do jogo compilado.

O trabalho permite observar o ganho de flexibilidade que o emprego de tais técnicas pode agregar ao design de jogos, bem como as limitações técnicas e que decorrem do crescimento de complexidade de modelagem à medida que se multiplicam as escolhas e contextos com os quais lidam os agentes.

Conclui-se que mesmo jogos de pequeno porte podem se beneficiar da
agregação de \emph{inteligência artificial} às interações entre jogadores
e \npc{}s.


\begin{comment}
Avanços tecnológicos permitem que cada vez mais detalhes e complexidade interajam em jogos.

Com frequência, IA é usada

Agentes Inteligentes. -> tópico da moda
“Agentes de software ou sistemas multi-agentes tem recentemente atraído uma considerável atenção da indústria de software. O objetivo deste capítulo é introduzir este novo paradigma de desenvolvimento que é adequado para o desenvolvimento de complexos sistemas de software. Debateremos porque que a abordagem orientada a agente é um genuíno avanço em relação ao estado da arte. Uma visão geral dos principais conceitos, métodos e ferramentas existentes é apresentada. Descreveremos em detalhe o framework Tropos. Os principais desafios do paradigma de agentes para o desenvolvimento de software são discutidos.
[ JAI ] - Livro das Jornadas de Atualização em Informática
2006 jul. 17-20 : Campo Grande - MS - ISBN 85-87926-19-5
descrição
SEGUNDO ISBN:
TÍTULO: 	ATUALIZAÇÕES EM INFORMÁTICA
ORGANIZADOR: 	KARIN BREITMAN
ORGANIZADOR: 	RICARDO ANIDO
Nº DE EDIÇÃO: 	1
ANO DE EDIÇÃO: 	2006
TIPO DE SUPORTE: 	PAPEL
PÁGINAS: 	464
EDITORA: 	PUC-RIO


”

Falar da motivação para uso de técnicas de IA em jogos, e do potencial
que o BDI e a técnica blackboard têm de agregar ao realismo da experiência
dos jogos. (Citar artigos falando do emprego dessas técnicas.)

Falar do intuito do estudo de caso, do que se pretende (e como)
quantificar. Mencionar as maiores escolhas e como imaginamos que elas
delinearam os resultados. (E instigar o leitor à leitura do restante
da tese!)


% Exemplo de resumo --- muito bom!
\begin{comment}
Numa série de artigos publicados entre 1843 e 1844, M.Hess sustenta
que a origem de um sistema de coordenadas espaço-temporais
singularmente compostas demonstra a irrefutabilidade das vantagens das
posturas dos filósofos divergentes com relação às suas
atribuições. Deve-se produzir um conceito que a forma de uma
transcendência imanente ou primordialassume importantes posições no
estabelecimento da lógica da aparência, psicologia racional,
cosmologia racional e, por fim, da teologia racional. Percebemos, cada
vez mais, que o mundo líquido em que vivemos facilita a criação da
determinação do Ser enquanto Ser. Todas estas questões, devidamente
ponderadas, levantam dúvidas sobre se o tríptico movimento de
pensamento nos obriga à análise da afirmação que o Ser é e o Não ser
não é. É importante questionar o quanto a expansão dos mercados
mundiais desafia a capacidade de equalização da fórmula da ressonância
racionalista. A prática cotidiana prova que a revolução copernicana,
entendida como ruptura, é um subconjunto do fundo comum da
humanidade. Um teórico da redundância negaria que o conceito platônico
de pólis ideal deve passar por modificações independentemente do
realismo ingênuo, isto é, da crença equivocada na confiabilidade dos
dados sensoriais transmitidos pela realidade fenomenal. Todavia, o
surgimento do comércio virtual auxilia a preparação e a composição dos
modos de análise convencionais. Ora, a complexidade dos estudos
efetuados não resulta em uma interiorização imanente do aparelho
repressivo, coercitivo, do sistema. Podemos já vislumbrar o modo pelo
qual o forte compromisso ontológico da teoria dos conjuntos limita as
atividades dos relacionamentos verticais entre as hierarquias
conceituais. De maneira sucinta, a interioridade do Ser social,
eminentemente enquanto Ser, prova que a mutação pós-jungiana
representa uma abertura para a melhoria do levantamento das variáveis
envolvidas.

\end{comment}
% Acho que em algum lugar me disseram que as palavras-chave devem vir
% do pessoal da biblioteca. Mas é possível que eu esteja só me
% confundindo com algum campo da ficha catalográfica...
Palavras-chave: Engenharia. Engenharia da Computação. Inteligência
Artificial. Arquitetura BDI. Sistema Blackboard. Jogos.
